\documentclass[12pt]{article}
\usepackage[utf8]{inputenc}
\usepackage[T1]{fontenc}
\usepackage{amsmath,amsfonts,amssymb}
\usepackage{graphicx}
\usepackage{a4wide}\title{Industrial project description (Customer demand forecasting)}
\date{}
\begin{document}
\maketitle

Chosen role: Analyst.

\section{Planning the industrial research project}

\begin{enumerate}
\item Goal of the project. ~--- The goal of the project is to develop an accurate and robust predictive model for customer demand forecasting.
\item Applied problem solved in the project. ~--- The applied problem is forecasting future customer demand based on historical sales data. The results can be illustrated through performance metrics such as error rates, visualized predictions, and comparisons between predicted and actual demand.
\item Description of historical measured data. ~--- The historical data consists of time series data with attributes such as dates, sales volume, product types, and other features like promotions and economic indicators. Algebraically, this can be structured as a multivariate time series or a matrix of features and target values for supervised learning.
\item Quality criteria. ~--- The quality of the model will be assessed using error functions such as RMSE, MAE or Huber Loss to achieve a balance between them.
\item Project feasibility. ~--- The project feasibility analysis begins with exploratory data analysis (EDA) to understand the distribution, trends, and potential outliers in the historical data. However, risks such as data quality issues, changing market conditions, and unforeseen external factors may impact the accuracy of the forecasts.
\item Conditions necessary for successful project implementation. ~--- The success of the project requires a large, high-quality dataset with minimal missing values, covering a significant time span to capture seasonality and trends. Additionally, external features such as economic factors and promotions must be captured to improve model accuracy.
\item Solution methods. ~--- The solution will involve a combination of statistical, machine learning, and possibly deep learning methods. Optimal methods may involve models like Random Forest, XGBoost, or neural networks, depending on the complexity of the data and the relationships it contains. Feature engineering, hyperparameter tuning, and model ensembling will also be key to improving model performance.
\end{enumerate}

\section{Research or development?}
In other words, novelty or technological advancement?

{Analyst:} What impact will the research have on the field of knowledge? How useful will it be?
\begin{itemize}
    \item The research will contribute to the knowledge of demand forecasting by enhancing the understanding of which models and features work best for predicting customer demand.
\end{itemize}


\end{document}