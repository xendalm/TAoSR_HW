\documentclass[12pt]{article}
\usepackage[utf8]{inputenc}
\usepackage[T1]{fontenc}
\usepackage{amsmath,amsfonts,amssymb}
\usepackage{graphicx}
\usepackage{a4wide}

\title{The Impact of Synthetic Data on Text Summarization Quality}
\date{}
\begin{document}

\maketitle

This project explores the role of synthetic data in enhancing the quality of automatic text summarization. Synthetic datasets, often generated using large language models, provide a promising solution to address data scarcity and improve model generalization. However, challenges such as domain-specific applicability and evaluation complexities remain. The project will design and implement synthetic data generation techniques, evaluating their integration into summarization workflows and measuring the impact on model performance.

\section{Introduction}

Table~\ref{tab:intro_comparative} provides a comparative analysis of existing solutions for automatic text summarization with a focus on the role of synthetic data.

\begin{table}[!htbp]
    \centering
    \caption{Comparative analysis of basic solution}
    \label{tab:intro_comparative}
    \begin{tabular}{p{5cm}|p{5cm}|p{5cm}}
        Solution                                                                                                                                                                                                & Strengths & Weakness \\
        \hline
        Synthetic data generation with LLMs for text classification~\cite{Li2023}                                                                                                                               &
        LLMs effectively generate synthetic datasets, especially in few-shot settings. Promising results for low-subjectivity tasks demonstrate its utility in specific domains.                                &
        Synthetic data struggles with high-subjectivity tasks and lacks diversity compared to real-world data. Dependence on real examples and high computational costs limit broader adoption.                                        \\
        \hline
        Using SFT and RLHF for summarization in Russian~\cite{Sber2024}                                                                                                                                         &
        Synthetic datasets improve summarization quality and address data scarcity in low-resource languages. Shows that synthetic data paired with RLHF, can align models more closely with human preferences. &
        High-quality synthetic data requires significant resources and struggles with real-world variability. Over-reliance on synthetic data can hinder generalization and practical scalability.                                     \\
    \end{tabular}
\end{table}

\nocite{*} % Remove this to keep the cited referernces only

\bibliographystyle{unsrt}
\bibliography{biblio}
\end{document}
